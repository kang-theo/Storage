\documentclass{article}
\usepackage[utf8]{inputenc}
\usepackage{quiver}
\usepackage{tikz-cd}

\begin{document}

\title{ quiverTest}

\author{Theo}

\date{March 26, 2022}

\maketitle

% https://q.uiver.app/?q=WzAsNCxbMCwwLCJhXjIiXSxbMSwwLCJiXzIiXSxbMSwxLCJcXGJ1bGxldCJdLFswLDEsIlxcYnVsbGV0Il0sWzAsMSwiIiwwLHsiY3VydmUiOi0xLCJjb2xvdXIiOlswLDYwLDYwXX1dLFswLDEsIiIsMix7ImN1cnZlIjoxLCJjb2xvdXIiOlsyNDAsNjAsNjBdfV0sWzEsMiwiZCIsMSx7ImN1cnZlIjotM31dLFsyLDNdLFszLDAsImMiLDEseyJjdXJ2ZSI6LTN9XV0=
\[\begin{tikzcd}
	{a^2} & {b_2} \\
	\bullet & \bullet
	\arrow[color={rgb,255:red,214;green,92;blue,92}, curve={height=-6pt}, from=1-1, to=1-2]
	\arrow[color={rgb,255:red,92;green,92;blue,214}, curve={height=6pt}, from=1-1, to=1-2]
	\arrow["d"{description}, curve={height=-18pt}, from=1-2, to=2-2]
	\arrow[from=2-2, to=2-1]
	\arrow["c"{description}, curve={height=-18pt}, from=2-1, to=1-1]
\end{tikzcd}\]

\end{document}
