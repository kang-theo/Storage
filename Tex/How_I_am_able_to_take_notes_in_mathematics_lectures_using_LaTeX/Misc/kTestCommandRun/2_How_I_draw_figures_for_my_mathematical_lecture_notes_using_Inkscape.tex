% \documentclass[12pt]{report}
\documentclass[12pt]{article}
\usepackage{import}
\usepackage{xifthen}
\usepackage{pdfpages}
\usepackage{transparent}
\usepackage{xcolor}
\usepackage{graphicx}
\usepackage{float}

% define 2 commands to insert figures quickly, need choose and inkscape installed.
% The settings assume the following directory structure:
% master.tex
% figures/
%     figure1.pdf_tex
%     figure1.svg
%     figure1.pdf
%     figure2.pdf_tex
%     figure2.svg
%     figure2.pdf
% pdf_tex and pdf are exported when saving diagrams as pdf.
% svg is exported when saving the inkscape project.

\newcommand{\incfig}[1]{%
    \def\svgwidth{\columnwidth}
    \import{./figures/}{#1.pdf_tex}
}

% use \incfig[0.3]{figure-name} to set the size of the figures.
\newcommand{\incfigsetwidth}[2][1]{%
    \def\svgwidth{#1\columnwidth}
    \import{./figures/}{#2.pdf_tex}
}

\begin{document}
% \pdfsuppresswarningpagegroup=1

% kTest inserting figures generated by inkscape
\begin{figure}[H]
    \centering
    \incfig{kTest_drawing}
    \caption{kTest inserting figures generated by inkscape}
    \label{fig:kTest_drawing}
\end{figure}

% kTest for sleepymalc/VSCode-LaTeX-Inkscape, a macOS way to implement sleepymalc/VSCode-LaTeX-Inkscape
% invoded by "ctrl+i i" defined in vscode keybindings.json
\begin{figure}[H]
    \centering
    \incfig{figure_test}
    \caption{Title for figure test}
    \label{fig:figure_test}
\end{figure}

\end{document}