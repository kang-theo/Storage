\documentclass[no-math, xcolor=table]{beamer} %no-math 不影响数学字体
% compress → 以紧凑方式显示导航栏内容
% handout → 取消重叠和动画效果, 便于打印

% %讲义,竖向每页2张
% \usepackage{pgfpages}
% \pgfpagesuselayout{2 on 1}[a4paper,border shrink=5mm]
% %横向,每页4张
% \usepackage{pgfpages}
% \pgfpagesuselayout{4 on 1}[a4paper,border shrink=5mm, landscape]

% 修正 XeTeX 下导航按钮失效问题
% \usepackage{patchbeamer}

\bibliographystyle{apalike}

% package import=======================
\usepackage[space, hyperref, UTF8]{ctex}

% 主题样式
% ◦ 没有导航条:default,AnnArbor,Bergen,Boadilla,CambridgeUS,... 
% ◦ 带树形导航条: Antibes, JuanLesPins, Montpellier
% ◦ 带侧边导航条:Berkeley,Goettingen,Hannover,Marburg,PaloAlto 
% ◦ 带微型导航条:Berlin,Ilmenau,Darmstadt,Dresden,Frankfurt,... 
% ◦ 带节小节标题: Copenhagen, Luebeck, Malmoe, Warsaw
% \usetheme{Warsaw}
% \usecolortheme{crane}
% \usetheme{PaloAlto}
\usetheme{CambridgeUS}
% \usetheme{Berlin}
% \usetheme{Singapore}
% \usetheme{Madrid}

% \usepackage{graphicx} %beamer 已自动包含了 graphicx 包,用于插入图片,也自动包含了 fontspec

\logo{\includegraphics[width=0.1\textwidth]{Adelaide.png}}
% \logo{NCCU} %会覆盖上面的图标

%字体格式设置
% \usepackage{mathpazo}
% \usefonttheme{serif} 
\usefonttheme[onlymath]{serif} %中文环境数学字体格式

%插图自动编号
% \usepackage[noindent,UTF8]{ctexcap}
%图、表显示编号:表1,图1等
\setbeamertemplate{caption}[numbered]
%不想显示页眉信息
\setbeamertemplate{headline}{}
%不显示页脚信息
\setbeamertemplate{footline}{}
%不显示下面导航栏
\setbeamertemplate{navigation symbols}{}


%每节之间加个目录
\setbeamerfont{myTOC}{series=\bfseries, size=\Large}
\AtBeginSection{\frame{\frametitle{Outline}%
				\usebeamerfont{myTOC}\tableofcontents[current]}}

%幻灯片计时
% \usepackage[timeinterval=15]{tdclock}
\usepackage[font=Times, timeinterval=15, timeduration=5, timewarningfirst=85, timewarningsecond=90, fillcolorwarningsecond=white!60!yellow]{tdclock}

%显示幻灯片时间
% \date[\initclock\mmddyyyy\tddate\ \ \hhmmss\tdtime]{\today}




% \author{kcy}
% \title{First Beamer Start}

\title[short title]{long long long title}
% \title[標題 \hspace{2em}\insertframenumber/\inserttotalfr {簡報標題}
\subtitle[short subtitle]{long subtitle}
\author[short author names]{long author names}
\date[short date]{long date}
\institute[short institute]{long institute name}
\titlegraphic{\includegraphics[width=0.35\textwidth]{Adelaide.png}}
%设置封面标题颜色大小
% \setbeamerfont{title}{size=\LARGE}
% \setbeamercolor{title}{fg=yellow,bg=gray}


% % begin tableofcontents format setting,from web
% \makeatletter
% \def\beamer@tocaction@only#1{\only<.(1)>{\usebeamertemplate**{#1}}}
% \define@key{beamertoc}{subsectionsonly}[]{\beamer@toc@subsectionstyle{show/only}\beamer@toc@subsubsectionstyle{show/shaded/hide}}

% \newif\ifbeamer@pausebeforesubsections
% \define@key{beamertoc}{pausebeforesubsections}[true]{\beamer@pausebeforesubsectionstrue}

% \patchcmd{\beamer@tableofcontents}{\beamer@pausesectionsfalse}%
%   {\beamer@pausesectionsfalse\beamer@pausebeforesubsectionsfalse}{}{}

% \patchcmd{\beamer@subsectionintoc}{\ifbeamer@pausesubsections\pause\fi}%
%   {\ifbeamer@pausesubsections\pause\else%
%    \ifbeamer@pausebeforesubsections\ifnumequal{#2}{1}{\pause}{}\fi\fi}{}{}
% \makeatother
% % end tableofcontents format setting 

% %1 在每节前加一个目录
% \AtBeginSection[]
% {
% 	\begin{frame}
% 		\tableofcontents[currentsection]
% 	\end{frame}
% }
% 2 在每节前插入目录
% \AtBeginSection[]{\frame{\tableofcontents[currentsection,hideallsubsections]}}

\AtBeginLecture{
	\begin{frame}
	\Large
	This lecture added: \insertlecture	
	\end{frame}
}

%生成节标题页
% \AtBeginSection{
%   \begin{frame}
%   \sectionpage
%   \end{frame}
% }
% \AtBeginSubsection{
% 	\begin{frame}
% 	\subsectionpage
% 	\end{frame}
% }

%设置每个页面背景颜色
% \setbeamertemplate{background}[grip][step=4mm,color=lightgray] %设置网格线
\setbeamercolor{normal text}{bg=lightgray}
%设置渐变背景色
\definecolor{bottomcolor}{rgb}{0.32, 0.3, 0,38}
\definecolor{middlecolor}{rgb}{0.08, 0.08, 0.16}
\setbeamertemplate{background canvas}[vertical shading][bottom=bottomcolor, middle=middlecolor, top=black]


\begin{document}

% %显示幻灯片时间,有问题
% \begin{frame}
% % \initclock
% Time now: \tdtime; Time past: \crono
% \end{frame}


%三种生成 titlepage 的方式
% 1
\frame{\titlepage}
%显示本页页码
% \insertframenumber/\inserttotalframenumber
% 2
% \begin{frame}
	% \maketitle
% \end{frame}
% 3
% \titlepage %不在 frame 中,默认在一张空白 frame 中显示 titlepage
% 4
% \begin{frame}
% \titlepage
% \end{frame}
% 5
% \begin{frame}[plain] %plain 风格
%   \titlepage
% \end{frame}

% %another title page
% \begin{frame}
	% \title{Another title page}
	% \author{kcy\\a1738097} 
	% \institute[short institute]{long institute name}
	% \titlegraphic{\includegraphics[width=0.35\textwidth]{Adelaide.png}}
	% \centering{a1738097}
	% \date{\today}
	% \maketitle 
% \end{frame}

% 目录
% % \tableofcontents的常用选项
% - pausesections → 逐次显示目录
% - hideallsubsections → 不显示所有子节标题
% - hideothersubsections → 不显示其它节的子节标题
% - currentsection → 仅显示当前节标题, 其它的以半透明方式显示 - currentsubsection → 显示当前节的子节标题
% - sections={m-n} → 只在目录中显示第 m 到第 n 节
\begin{frame}\frametitle{Outline}
% \begin{frame}{Outline}{suboutline} %another way to generate frame, subtitle can also be added
	\tableofcontents[pausesections] %这个必须放在一个 frame中,这里是放在 Outline frame 中, pausesections 是分步展现内容
	% \tableofcontents[subsectionsonly, pausesections]
	% \tableofcontents[subsectionsonly, pausebeforesubsections, pausesections]
	% \tableofcontents[hideothersubsections, pausesections]
	% \tableofcontents[hideallsubsections] %不显示子章节	

	%设置目录字体颜色和大小
	\setbeamercolor{section in toc}{fg=yellow!80!gray}
	\setbeamertemplate{section in toc}[sections numbered]
	%其中第一行将节标题颜色设为 yellow!80!gr 第二行设定在目录中显示节标题编号.页中节标题的模板和颜色:

\end{frame}

%Another table of contents
% \frame{\tableofcontents}
% \frame{\tableofcontents[currentsection]} %高亮显示当前讲述章节


\lecture{lecture name\_1}{lecture label\_1}


\part{Part}
\begin{frame}
	\partpage
\end{frame}


%比 part 更高一级,统一多个人的讲稿在一个 pdf 文件中,或者个人的多次讲稿在一个文件中
%涉及到导言区的设置
\lecture{lecture name\_2}{lecture label\_2}


\begin{frame}
\sectionpage
\end{frame}

\section{First section}
	\subsection{First subsection}
	\begin{frame}[c]\frametitle{First frame}
	The counter goes here
\end{frame}

\section{Second section}
	\subsection{Second subsection}
	\begin{frame}[c]\frametitle{Second frame}
	contents in second section
\end{frame}



\section*{Third section} % 不显示在目录中
	\subsection{Third subsection}
	\begin{frame}[plain]

	This is the plain frame, it may be used to show a big picture

	\end{frame}



%列表
\section{Fourth section}
\begin{frame}[t]\frametitle{itemize}
	\begin{itemize}
		\item The first item
		\item The second item
		\item The third item
	\end{itemize}
\end{frame}

\begin{frame}[t]\frametitle{enumerate}
	\begin{enumerate}
		\item The first item
		\item The second item
		\item The third item
	\end{enumerate}
\end{frame}

\begin{frame}[t]\frametitle{description}
	\begin{description}
		\item[First Item] Description of first item
		\item[Second Item] Description of second item
		\item[Third Item] Description of third item
	\end{description}
\end{frame}

% 插入图片
% \begin{frame}[c]{picture}{picture insert}
% 	\begin{figure}
% 		\centering
% 		\includegraphics[width=.8\textwidth]{Wireless_network_topology.jpg} %这里文件名必须不能有空格
% 		\caption{Wired connection topology(Cisco, 2017)}
% 	\end{figure}
% \end{frame}

\begin{frame}[c]\frametitle{multi-columns}
	%多栏环境
	\begin{columns}
		\column{.49\textwidth} %版心的0.49倍
		\begin{itemize}
			\item The first item
			\item The second item
			\item The third item
		\end{itemize}
		\column{.49\textwidth}
		\begin{enumerate}
			\item The first item
			\item The second item
			\item The third item
		\end{enumerate}
	\end{columns}

\end{frame}

% 自动分帧	
\begin{frame}[t, allowframebreaks]\frametitle{title}
	afea\\
	sfjaf∫\\
	lsfjls\\
	ljfs\\
	lsf\\
	ljfslj\\
	sdlf\\
	sfj\\
	sdfj\\
	sdfjl\\
	ljfs\\
	lsf\\
	ljfs\\
	lsf\\
	ljfslj\\
	sdlf\\
	sfj\\
	sdfj\\
	sdfjl\\
	ljfs\\
	ljfslj\\
	sdlf\\
	sfj\\
	sdfj\\
	sdfjl
\end{frame}

% 缩小比例,应用在每一页只多一点的情况下,最多不要縮小超過 5%
\begin{frame}[shrink=5]\frametitle{title}
	afea\\
	sfjaf∫\\
	lsfjls\\
	ljfs\\
	lsf\\
	ljfslj\\
	sdlf\\
	sfj\\
	sdfj\\
	sdfjl\\
	ljfs\\
	lsf\\
	ljfs\\
	lsf\\
	ljfslj\\
	sdlf\\
	sdfjl
\end{frame}


%小标题着色 block 环境
\begin{frame}{自如之理,乃见真实}
	\begin{block}{佛告须菩提}
	凡所有相,皆是虚妄。若见诸相非相,则见如来。
	\end{block}

	\begin{alertblock}{佛告须菩提}
	凡所有相,皆是虚妄。若见诸相非相,则见如来。
	\end{alertblock}

	\begin{exampleblock}{佛告须菩提}
	凡所有相,皆是虚妄。若见诸相非相,则见如来。
	\end{exampleblock} 
\end{frame}

%特效:分步展现
\begin{frame}{受持此经,功德无量} 
	\begin{itemize}
		\item 初日分以恒河沙等身布施 \pause
		\item 中日分复以恒河沙等身布施 \pause
		\item 后日分亦以恒河沙等身布施 \pause
		\item 如是无量百千万亿劫以身布施
	\end{itemize} 
\end{frame}

%分步展现:+号相当于\pause,-相当于1-、2-等,在 item 环境后加上[<+->],相当于在每个 item 后面加上[<+->]
\begin{frame}{受持此经,功德无量} 
	\begin{itemize}[<+->]
		\item 初日分以恒河沙等身布施 
		\item 中日分复以恒河沙等身布施
		\item 后日分亦以恒河沙等身布施 
		\item 如是无量百千万亿劫以身布施
	\end{itemize} 
\end{frame}

% <+-| alert@+> <+-| structure@+>
\begin{frame}{古希腊数学}
勾股定理在西方称为毕达哥拉斯定理,古希腊数学家在 2000 多年前就已经发现并证明了它。\pause
	\begin{itemize}
		\item<+-| alert@+>
		  公元前 6 世纪,毕达哥拉斯学派发现一个法则,可以构造直角三角形的边长;
		\item<+-| alert@+>
		  公元前 3 世纪,欧几里德《几何原本》使用面积法证明勾股定理。
	\end{itemize}
\end{frame}

%特效:分步展现之精确控制展现步骤
\begin{frame}{受持此经,功德无量} 
	\begin{itemize}
		\onslide<2>{\item 初日分以恒河沙等身布施} 
		\only{\item only}
		\onslide<3->{\item 中日分复以恒河沙等身布施} %3步及之后都展现
		\onslide<2,4>{\item 后日分亦以恒河沙等身布施} %2、4步展现
		\onslide<1>{\item 如是无量百千万亿劫以身布施}

	\end{itemize} 
\end{frame}

%分步展现:+号相当于\pause,-相当于1-、2-等,在 item 环境后加上[<+->],相当于在每个 item 后面加上[<+->]
\begin{frame}{受持此经,功德无量} 
	\begin{itemize}[<+->]
		\onslide<2>{\item 初日分以恒河沙等身布施} 
		\only{\item only}
		\onslide<3->{\item 中日分复以恒河沙等身布施} %3步及之后都展现
		\onslide<2,4>{\item 后日分亦以恒河沙等身布施} %2、4步展现
		\onslide<1>{\item 如是无量百千万亿劫以身布施}

	\end{itemize} 

	% \uncover 控制内容什么时候显示, 但不显示时仍占据版面, \only 内容不显示时完全消失(不占版面)
	\only<4>{only 第4张以后才会出现} 

	only 测试

	\uncover<6>{uncover 第6张以后 才会出现}

	uncover 测试
\end{frame}

\begin{frame}{古中国数学}{定理证明}
有论者认为早在公元前 11 世纪商高即已证明勾股定理。完整的证明见于三国时(公元 3 世纪)赵爽对《周髀算经》的注释。
	\pause
	\begin{figure}
		\centering
		\includegraphics[height=0.4\textheight]{Adelaide.png}
		\caption{勾股图}
	\end{figure}
\end{frame}

\begin{frame}
  \frametitle{背景介绍}
\begin{itemize}
\item 考虑问题
    $$ a^2+b^2=c^2.$$

\bigskip %\smallskip
\item 问题应用背景
    \begin{itemize}
        \item  xxxxx
        \item  xxxx
        \item  xxxxx
        \item $\cdots\ \cdots$  %省略号
    \end{itemize}
\end{itemize}
\end{frame}

% 强调
\begin{frame}{强调}
	最重要的就是 \alert{這一點}
	只有在 \alert<2>{第二張} 才重要。

	% 绿色强调
	{\color{green}{綠色的文字}}, 其他正常。
	只有在 {\color<2>{green}{第二張}} 才是綠色的。

	\begin{block}{小重點} 
		重點就是重點。 
	\end{block}

	\begin{alertblock}{大重點} 
		特別重要的東西。 
	\end{alertblock}
\end{frame}

% 显示源代码,并进行转义
\begin{frame}{源代码}

	% \begin{verbatim} 
	% 	for i in range(10):
	% 	print i 	
	% \end{verbatim}

	\begin{semiverbatim} 
		for \alert{i} in range(10):
		print \alert{i}
	\end{semiverbatim}

\end{frame}

\begin{frame}[fragile]
\frametitle{An Algorithm For Finding Primes Numbers.}
\begin{verbatim}
int main (void)
{
std::vector<bool> is_prime (100, true);
for (int i = 2; i < 100; i++)
if (is_prime[i])
{
std::cout << i << " ";
for (int j = i; j < 100; is_prime [j] = false, j+=i);
}
return 0;
}
\end{verbatim}

\begin{uncoverenv}<2> 
	Note the use of \verb|std::|. 
\end{uncoverenv}

\end{frame}


%幻灯片跳转
\begin{frame}[label=here]{幻灯片跳转}
	過來這裡!
\end{frame}

\begin{frame}{幻灯片跳转}
	\hyperlink{here}{\beamerbutton{去吧}}
\end{frame}


% \begin{frame}{勾股数}
% 	\begin{table}
% 		\centering
% 		% 颜色 craneorange 是在 crane 色彩主题中定义的
% 		\rowcolors{2}{craneorange!25}{craneorange!50} %需要在文档类中开启 xcolor 选项, 这里的 cranneorange 颜色需要使用 crane颜色主题
% 		\begin{tabular}{rrr}
% 			\rowcolor{craneorange}直角边 $a$ & 直角边 $b$ & 斜边 $c$\\
% 			3 & 4 & 5 \\
% 			5 & 12 & 13 \\
% 			7 & 24 & 25 \\
% 			8 & 15 & 17 \\
% 		\end{tabular}
% 		\caption{较小的几组勾股数}
% 	\end{table}
% \end{frame}

% \begin{frame}{带色边框}
% %彩色盒子
% % % 常用选项
% % - wd → 盒子宽度, 缺省为 wd=\textwidth
% % - colsep → 文本与盒子边界的间距
% % - colsep* → 文本与盒子上下边界的间距
% % - rounded=true → 圆角, 缺省为直角
% % - shadow=true → 添加阴影, 增加立体感, 需与 rounded 一起使用 - center → 文本与盒子居中对齐, 缺省为左对齐
% 	\begin{beamercolorbox}[wd=2\textwidth, rounded=true, shadow=true, center]{red} 
% 	% \begin{beamercolorbox}{red} 
% 	aslfjlsdafsdlaf
% 	\end{beamercolorbox}
% \end{frame}

 %带标题的圆角盒子
%  常用选项
% - upper=beamer 颜色 → 指定标题区域的前景与背景颜色 - lower=beamer 颜色 → 指定文本区域的前景与背景颜色 - width → 盒子宽度
% - shadow=true → 添加阴影, 增加立体感
 % \begin{beamerboxesrounded}[选项]{标题} 
 % ... ...
 % \end{beamerboxesrounded}


% %添加参考文献
% \begin{frame}[allowframebreaks]{References}
	% \nocite{keylist}
	% \def\newblock{}
	% \bibliographystyle{plain}
	% \bibliography{mybib.bib}
% \end{frame}

\begin{thebibliography}{10} %这里9 或者99表示文献列表最大值
\bibitem{Golub96}
 G.H. Golub and C. Van loan,
 \newblock ‘‘Matrix Computations,’’
 \newblock The Jhon Hopkins University press, 1996.
\end{thebibliography}

% ◦ plain→取消headlines,footlines和sidebars
% ◦ t,c,b→文字竖直方向的对齐方式
% ◦ fragile/containsverbatim→若需要使用抄录环境,则应该加该选项 ◦ shrink→自动缩小,以便放下所有内容
% ◦ allowframebreaks→当内容太长时,允许分帧显示
\begin{frame}[c,plain]
\begin{center}
\Huge\color{red}\heiti\bfseries 谢\quad 谢!

  Thank you!
\end{center}
\end{frame}

\end{document}