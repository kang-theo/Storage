\documentclass{ctexart}
\usepackage{markdown}
\usepackage[colorlinks]{hyperref}
\markdownSetup{rendererPrototypes={
 link = {\href{#2}{#1}}
}}

\begin{document}

% method 1
\begin{markdown}

# 第一章 欢迎学习玩转数据结构 
## 1-1 欢迎学习《玩转数据结构》

数据结构研究的是数据如何在计算机中进行组织和存储,使得我们可以**高效**的获取数据或者修改数据

分类:
1. 线性结构:
    * 数组
    * 栈
    * 队列
    * 链表
    
1. 树结构:
    * 二叉树
    * 二分搜索树
    * AVL
    * 红黑树
    * Treap
    * Splay
    * 堆
    * Trie
    * 线段树
    * K-D 树
    * 并查集
    * 哈弗曼树
    * ... ...
    
1. 图结构:
    * 邻接矩阵
    * 邻接表

数据结构例子:
1. 数据库:  ---> 一个软件 ---> 需要底层很多数据结构:树,哈希表等

2. 操作系统:多任务切换
    1. 系统栈:递归调用
    2. 优先队列:堆
        * 在多任务间比较优先级,以便进行任务切换
1. 文件压缩:哈夫曼树
2. 通讯录:Trie - 前缀树(替换链表结构-查找速度慢)

3. 游戏:寻路算法
    * 图论算法
        DFS:使用栈
        BFS:使用队列
        
**课程设置**:
Based on Java

||课程设置||
|---|---|---|
|数组(基础)|二分搜索树(基础)|并查集(竞赛)|
|栈(基础)|堆(基础)|AVL(平衡二叉树、复杂,代码量稍大)|
|队列(基础)|线段树(竞赛)|红黑树(平衡二叉树、复杂,代码量稍大)|
|链表(基础)|Trie(竞赛)|哈希表|

不包含图:图结构使用简单的线性表就可以存储,但是图论领域以算法为主。

课程使用:**LeetCode题库**

不仅单单实现,还会进行优化,揭示数据结构背后的思考。

## 1-2 学习数据结构(和算法)到底有没有用

现状:门槛越来越低,开发工具通过接口提供了数据结构和算法,开发者使用即可
搭建上述开发工具和开发框架涉及大量数据结构
![picture](达拉然封家.png)

学好数据结构可以提升技术上限,在计算机科学(计算机技术)道路上走的更远


\end{markdown}

% method 2
% \markdownInput{/Users/kangchangyu/Downloads/第二章_不要小瞧数组.md}
\markdownInput{第二章_不要小瞧数组.md}

\end{document}
