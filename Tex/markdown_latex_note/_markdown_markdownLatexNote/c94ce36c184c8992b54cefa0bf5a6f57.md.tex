\markdownRendererHeadingOne{第二章 不要小瞧数组}\markdownRendererInterblockSeparator
{}\markdownRendererHeadingTwo{2-1 使用 Java 中的数组}\markdownRendererInterblockSeparator
{}数组三种定义方式: 1. int[] arr = new int[10]; 之后使用 for 循环为每个元素赋初值 2. int[] arr = new int[]\markdownRendererLeftBrace{}100, 99, 66\markdownRendererRightBrace{}; 3. int[] arr = \markdownRendererLeftBrace{}100, 99, 66\markdownRendererRightBrace{} 注:1、2中分配空间在堆中,3分配空间在栈中\markdownRendererInterblockSeparator
{}\markdownRendererHeadingTwo{2-2 二次封装属于我们自己的数组}\markdownRendererInterblockSeparator
{}\markdownRendererStrongEmphasis{数组优点:}随机读取,快速查询,所以数组最好应用于“索引有语意”的情况,比如索引表示学号,那么 scores[2]就是获取学号为2的同学的分数。\markdownRendererInterblockSeparator
{}但并非都是“索引有语意”最好,比如使用身份证号查询某人的工资,那么如果以身份证作索引,就要开辟很大的数组空间,其中很大一部分是浪费的。此时可以通过一个函数对这个初始索引做进一步处理,三列在一个范围内,并且尽量避免重复,然后把这个函数的结果作为索引--类似哈希表(散列表)\markdownRendererInterblockSeparator
{}数组没有语意的情况下,会有以下一些问题: 1. 如何表示没有元素? 2. 如何添加和删除元素? 3. ... ...\markdownRendererInterblockSeparator
{}```Java public class Array\markdownRendererLeftBrace{}\markdownRendererInterblockSeparator
{}\markdownRendererInputVerbatim{./_markdown_markdownLatexNote/bc245194db0eea0f6460618ef10f80ca.verbatim}\markdownRendererInterblockSeparator
{}\markdownRendererRightBrace{} ```\relax