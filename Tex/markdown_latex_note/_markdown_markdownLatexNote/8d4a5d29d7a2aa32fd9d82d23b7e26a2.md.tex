\markdownRendererHeadingOne{第一章 欢迎学习玩转数据结构}\markdownRendererInterblockSeparator
{}\markdownRendererHeadingTwo{1-1 欢迎学习《玩转数据结构》}\markdownRendererInterblockSeparator
{}数据结构研究的是数据如何在计算机中进行组织和存储,使得我们可以\markdownRendererStrongEmphasis{高效}的获取数据或者修改数据\markdownRendererInterblockSeparator
{}分类: 1. 线性结构: * 数组 * 栈 * 队列 * 链表\markdownRendererInterblockSeparator
{}\markdownRendererOlBegin
\markdownRendererOlItemWithNumber{1}树结构:\markdownRendererInterblockSeparator
{}\markdownRendererUlBeginTight
\markdownRendererUlItem 二叉树\markdownRendererUlItemEnd 
\markdownRendererUlItem 二分搜索树\markdownRendererUlItemEnd 
\markdownRendererUlItem AVL\markdownRendererUlItemEnd 
\markdownRendererUlItem 红黑树\markdownRendererUlItemEnd 
\markdownRendererUlItem Treap\markdownRendererUlItemEnd 
\markdownRendererUlItem Splay\markdownRendererUlItemEnd 
\markdownRendererUlItem 堆\markdownRendererUlItemEnd 
\markdownRendererUlItem Trie\markdownRendererUlItemEnd 
\markdownRendererUlItem 线段树\markdownRendererUlItemEnd 
\markdownRendererUlItem K-D 树\markdownRendererUlItemEnd 
\markdownRendererUlItem 并查集\markdownRendererUlItemEnd 
\markdownRendererUlItem 哈弗曼树\markdownRendererUlItemEnd 
\markdownRendererUlItem ... ...\markdownRendererUlItemEnd 
\markdownRendererUlEndTight \markdownRendererOlItemEnd 
\markdownRendererOlItemWithNumber{2}图结构:\markdownRendererInterblockSeparator
{}\markdownRendererUlBeginTight
\markdownRendererUlItem 邻接矩阵\markdownRendererUlItemEnd 
\markdownRendererUlItem 邻接表\markdownRendererUlItemEnd 
\markdownRendererUlEndTight \markdownRendererOlItemEnd 
\markdownRendererOlEnd \markdownRendererInterblockSeparator
{}数据结构例子: 1. 数据库: ---> 一个软件 ---> 需要底层很多数据结构:树,哈希表等\markdownRendererInterblockSeparator
{}\markdownRendererOlBegin
\markdownRendererOlItemWithNumber{2}操作系统:多任务切换\markdownRendererInterblockSeparator
{}\markdownRendererOlBeginTight
\markdownRendererOlItemWithNumber{1}系统栈:递归调用\markdownRendererOlItemEnd 
\markdownRendererOlItemWithNumber{2}优先队列:堆\markdownRendererInterblockSeparator
{}\markdownRendererUlBeginTight
\markdownRendererUlItem 在多任务间比较优先级,以便进行任务切换\markdownRendererUlItemEnd 
\markdownRendererUlEndTight \markdownRendererOlItemEnd 
\markdownRendererOlEndTight \markdownRendererOlItemEnd 
\markdownRendererOlItemWithNumber{3}文件压缩:哈夫曼树\markdownRendererOlItemEnd 
\markdownRendererOlItemWithNumber{4}通讯录:Trie - 前缀树(替换链表结构-查找速度慢)\markdownRendererOlItemEnd 
\markdownRendererOlItemWithNumber{5}游戏:寻路算法\markdownRendererInterblockSeparator
{}\markdownRendererUlBeginTight
\markdownRendererUlItem 图论算法 DFS:使用栈 BFS:使用队列\markdownRendererUlItemEnd 
\markdownRendererUlEndTight \markdownRendererOlItemEnd 
\markdownRendererOlEnd \markdownRendererInterblockSeparator
{}\markdownRendererStrongEmphasis{课程设置}: Based on Java\markdownRendererInterblockSeparator
{}\markdownRendererPipe{}\markdownRendererPipe{}课程设置\markdownRendererPipe{}\markdownRendererPipe{} \markdownRendererPipe{}---\markdownRendererPipe{}---\markdownRendererPipe{}---\markdownRendererPipe{} \markdownRendererPipe{}数组(基础)\markdownRendererPipe{}二分搜索树(基础)\markdownRendererPipe{}并查集(竞赛)\markdownRendererPipe{} \markdownRendererPipe{}栈(基础)\markdownRendererPipe{}堆(基础)\markdownRendererPipe{}AVL(平衡二叉树、复杂,代码量稍大)\markdownRendererPipe{} \markdownRendererPipe{}队列(基础)\markdownRendererPipe{}线段树(竞赛)\markdownRendererPipe{}红黑树(平衡二叉树、复杂,代码量稍大)\markdownRendererPipe{} \markdownRendererPipe{}链表(基础)\markdownRendererPipe{}Trie(竞赛)\markdownRendererPipe{}哈希表\markdownRendererPipe{}\markdownRendererInterblockSeparator
{}不包含图:图结构使用简单的线性表就可以存储,但是图论领域以算法为主。\markdownRendererInterblockSeparator
{}课程使用:\markdownRendererStrongEmphasis{LeetCode题库}\markdownRendererInterblockSeparator
{}不仅单单实现,还会进行优化,揭示数据结构背后的思考。\markdownRendererInterblockSeparator
{}\markdownRendererHeadingTwo{1-2 学习数据结构(和算法)到底有没有用}\markdownRendererInterblockSeparator
{}现状:门槛越来越低,开发工具通过接口提供了数据结构和算法,开发者使用即可 搭建上述开发工具和开发框架涉及大量数据结构 \markdownRendererImage{-w683}{media/15343028063905/15343052490665.jpg}{media/15343028063905/15343052490665.jpg}{}\markdownRendererInterblockSeparator
{}学好数据结构可以提升技术上限,在计算机科学(计算机技术)道路上走的更远\relax