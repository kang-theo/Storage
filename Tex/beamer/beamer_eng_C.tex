\documentclass[10pt,compress,t,noamsthm,notheorem,table,handout]{beamer}
% ,table,handout
\usetheme{Boadilla}
\useinnertheme{circles}
\useoutertheme{shadow}
\usecolortheme{seahorse}
\usefonttheme[onlymath]{serif}
\setbeamertemplate{navigation symbols}{}
% \setbeamercovered{transparent}

\setbeamercolor{myfootline}{bg=white,fg=blue}
\definecolor{myfoot}{rgb}{0.5,0.2,0.5}
\definecolor{darkblue}{rgb}{0.1,0,0.85}
\setbeamertemplate{headline}
  { \leavevmode\begin{beamercolorbox}[wd=\paperwidth,ht=1.25ex,dp=1ex,left]{}
    \end{beamercolorbox}}
\setbeamertemplate{footline}% 自定义页脚
  { \leavevmode\mbox{%
    \begin{beamercolorbox}[wd=.75\paperwidth,ht=2.25ex,dp=1ex,left]{myfootline}%
        \rule{2em}{0pt}\color{myfoot}\ttfamily\scriptsize%
        %\insertshortauthor~(\insertshortinstitute)
    \end{beamercolorbox}%
    \begin{beamercolorbox}[wd=.25\paperwidth,ht=2.25ex,dp=1ex,right]{myfootline}%
       {\color{myfoot}\ttfamily\scriptsize\insertframenumber{}/%
        \inserttotalframenumber\hspace*{3ex}}
    \end{beamercolorbox}}
    \vskip0pt }

\setbeamercolor{frametitle}{fg=blue,bg=white}
\setbeamertemplate{frametitle}{%
  \leavevmode\linespread{1}\large\textbf{\insertframetitle}\par
  \color{structure.fg!30!bg}\rule[6pt]{\linewidth}{2pt}\par\vspace{-1.0em}}

% \setbeamertemplate{blocks}[default] % beamer块(含定理类环境)不要阴影
\setbeamertemplate{bibliography entry title}{}{}
\setbeamertemplate{bibliography entry location}{}{}
\setbeamertemplate*{bibliography entry note}{}{}
\setbeamersize{text margin left=0.75cm, text margin right=0.75cm}

\setbeamercolor{bluebox}{fg=black,bg=blue!10}
\setbeamercolor{redbox}{fg=black,bg=red!10}
\newenvironment{Boxblue}[1][\textwidth]
  {\begin{beamercolorbox}[sep=0.1em,shadow=true,wd=#1,rounded=true,center]{bluebox}}
  {\end{beamercolorbox}}
\newenvironment{Boxred}[1][\textwidth]
  {\begin{beamercolorbox}[sep=0.1em,shadow=true,wd=#1,rounded=true,center]{redbox}}
  {\end{beamercolorbox}}

\usepackage{amsmath,amssymb,amsfonts,bm}
\usepackage{graphicx,xcolor}
\graphicspath{{figures/}}
\usepackage{hyperref}
\hypersetup{pdfborder=001,colorlinks=true,linkcolor=darkblue,urlcolor=blue}
\usepackage{bbding}
\newcommand{\Bullet}{{\fontsize{6pt}{6pt}\selectfont\CircleSolid}}
\newcommand{\Hand}{{\fontsize{8pt}{6pt}\selectfont\HandRight}}
\newcommand{\zhu}{{\color{blue!40}\Bullet}}
\newcommand{\zhuu}{{\color{red!80}\Hand}}
\newcommand{\labeli}{\zhu}
\newenvironment{blist}%
    {\begin{list}{{\hfill\raisebox{1.12pt}{\color{blue!60}\zhu}}}{%
     \leftmargin2em\labelwidth1.5em\labelsep0.5em
     \itemsep1ex\itemindent0pt\parsep0pt\topsep0pt}}
    {\end{list}}
\newenvironment{myitem}
  {\begin{list}{{\hfill\raisebox{0pt}{\labeli}}}{%
    \setlength{\leftmargin}{1.2em}\labelwidth0.8em\labelsep.4em%
    \itemsep1ex\parsep2pt\itemindent0pt\topsep0pt}}{\end{list}}
\newenvironment{subitem}
  {\begin{list}{{\hfill\raisebox{0pt}{-}}}{%
    \setlength{\leftmargin}{1.2em}\labelwidth0.8em\labelsep.4em%
    \itemsep0ex\parsep2pt\itemindent0pt\topsep0pt}}{\end{list}}
\usepackage{colortbl}
\usepackage{tikz}
\usetikzlibrary{arrows}
\usepackage{stmaryrd}
\usepackage{algorithm,algpseudocode,caption}
\usepackage{booktabs}
\usepackage[framemethod=tikz]{mdframed}
\newmdenv[linecolor=green,middlelinewidth=1pt,%
          roundcorner=3pt,backgroundcolor=white,%
          innertopmargin=0.8em,innerbottommargin=0.5em,%
          innerleftmargin=3pt,innerrightmargin=3pt,%
          skipbelow=0.5em,skipabove=1em,%
          splittopskip=\topskip]{Block}
\newmdenv[linecolor=green,middlelinewidth=1pt,%
          roundcorner=3pt,backgroundcolor=red!5!white,%
          innertopmargin=0.5em,innerbottommargin=0.5em,%
          innerleftmargin=3pt,innerrightmargin=3pt,%
          skipbelow=0.5em,skipabove=1em,%
          splittopskip=\topskip]{redbox}
\newmdenv[linecolor=green,middlelinewidth=0.5pt,%
          %outerlinewidth=0.5pt,skipabove=0pt,
          roundcorner=3pt,backgroundcolor=white,%
          innerbottommargin=3pt,innerrightmargin=5pt,%
          innerleftmargin=5pt,leftmargin=0ex]{mathbox}
\newmdenv[linecolor=blue!5!green,middlelinewidth=0.5pt,%
          roundcorner=3pt,backgroundcolor=yellow!5,%
          % frametitle={Hello},frametitlebackgroundcolor=green!50,%
          % skipabove=2pt,skipbelow=2pt,%
          innerleftmargin=3pt,leftmargin=0ex]{notebox}
\newmdenv[linecolor=white,font={\scriptsize},%
          fontcolor=blue!85,backgroundcolor=yellow!5,%
          skipabove=1ex,skipbelow=0pt,innerbottommargin=0.5ex,%
          innerleftmargin=3pt,leftmargin=1em]{myref}

%%%%%%%%%%%%%%%%%%%%%%%%%%%%%%%%%%%%%%%%%%%%%%%%%%%%%%%%%%%%%%%%%%%%%%%%%%%%%%
\renewcommand{\thefootnote}{}% 不要编号
\setbeamertemplate{footnote}{% 首行不缩进
  \noindent\insertfootnotemark%
  \scriptsize\color{blue!85!green!85}\insertfootnotetext\par\kern1ex}
\renewcommand\footnoterule%    更改横线属性:长度,粗细,颜色
  {\color{red}\kern-3pt\rule{0.4\linewidth}{0.5pt}\par\kern2.6pt}
%%%%%%%%%%%%%%%%%%%%%%%%%%%%%%%%%%%%%%%%%%%%%%%%%%%%%%%%%%%%%%%%%%%%%%%%%%%%%%

%%===== 定理环境
\usepackage[amsmath,thref,thmmarks,hyperref]{ntheorem}
\theorempreskipamount1.2em  % spacing before the environment
\theorempostskipamount0em % spacing after the environment
%\theorempostwork{\noindent}
\theoremstyle{nonumberbreak}%{nonumberplain}
\theoremheaderfont{\bfseries\color{blue}}
\theorembodyfont{\normalfont}
\theoremindent0em
\theoremseparator{\hspace{0.2em}}
\theoremnumbering{arabic}
\colorlet{thmcolor}{green}
\newmdtheoremenv[linecolor=thmcolor,middlelinewidth=0.5pt,
    roundcorner=3pt,backgroundcolor=white,%
    innertopmargin=0.8em,innerbottommargin=0.5em,%
    innerleftmargin=3pt,innerrightmargin=3pt,%
    skipbelow=0.5em,skipabove=1em,%
    splittopskip=\topskip,ntheorem]{theorem}%
    {Theorem}
\newmdtheoremenv[linecolor=thmcolor,middlelinewidth=0.5pt,
    roundcorner=3pt,backgroundcolor=white,%
    innertopmargin=0.8em,innerbottommargin=0.5em,%
    innerleftmargin=3pt,innerrightmargin=3pt,%
    skipbelow=0.5em,skipabove=1em,%
    splittopskip=\topskip,ntheorem]{corollary}%
    {Corollary}
\newmdtheoremenv[linecolor=thmcolor,middlelinewidth=0.5pt,
    roundcorner=3pt,backgroundcolor=white,%
    innertopmargin=0.8em,innerbottommargin=0.5em,%
    innerleftmargin=3pt,innerrightmargin=3pt,%
    skipbelow=0.5em,skipabove=1em,%
    splittopskip=\topskip,ntheorem]{lemma}%
    {Lemma}
\newmdtheoremenv[linecolor=thmcolor,middlelinewidth=0.5pt,
    roundcorner=3pt,backgroundcolor=white,%
    innertopmargin=0.8em,innerbottommargin=0.5em,%
    innerleftmargin=3pt,innerrightmargin=3pt,%
    skipbelow=0.5em,skipabove=1em,%
    splittopskip=\topskip,ntheorem]{definition}%
    {Definition}
\newmdtheoremenv[linecolor=thmcolor,middlelinewidth=0.5pt,
    roundcorner=3pt,backgroundcolor=white,%
    innertopmargin=0.8em,innerbottommargin=0.5em,%
    innerleftmargin=3pt,innerrightmargin=3pt,%
    skipbelow=0.5em,skipabove=1em,%
    splittopskip=\topskip,ntheorem]{example}%
    {Example}

\usepackage[many]{tcolorbox}
\tcbset{highlight math %
  style={enhanced, colframe=blue!40,colback=yellow!20,arc=4pt,boxrule=1pt}}
\newtcbox{\subsubtit}[1][]{%
  after skip=1em,boxrule=0.5pt,
  fontupper=\color{blue}\bfseries,top=0.5ex,bottom=0.5ex,
  left=1ex,right=1ex,
  colframe=green,colback=red!5!white,#1}

%\renewcommand{\baselinestretch}{1.1}
\linespread{1.1}
\setlength{\parskip}{1ex}

\newcommand{\myem}[2][blue]{{\color{#1} #2}}
%\newcommand{\myref}[1]{\textcolor{blue}{\scriptsize #1}}

%%%%%========自定义命令=======================================
\newcommand{\lam}{\lambda}
\newcommand{\eps}{\varepsilon}
\newcommand{\dis}{\displaystyle}
\newcommand{\mycite}[1]{\textcolor{red}{\normalfont\upshape{#1}}}
\newcommand{\bbm}{\begin{bmatrix}}
\newcommand{\ebm}{\end{bmatrix}}

\renewcommand{\C}{\mathbb{C}}
\newcommand{\R}{\mathbb{R}}
\newcommand{\A}{\mathcal{A}}
\newcommand{\I}{\mathcal{I}}
\newcommand{\zp}{\{0\}}
\newcommand{\al}{\alpha}
\newcommand{\be}{\beta}
\newcommand{\ga}{\gamma}
\newcommand{\ep}{\varepsilon}
\newcommand{\si}{\sigma}
\newcommand{\ph}{\varphi}
\newcommand{\ry}{\forall}
\newcommand{\cz}{\exists}
\newcommand{\htt}{\Leftrightarrow}
\newcommand{\tc}{\Rightarrow}

\renewcommand{\rm}{\mathrm}

\newcommand{\xb}{\bar{x}}
\newcommand{\db}{\bar{\delta}}
\newcommand{\td}{\nabla}
\newcommand{\fh}{\hat{f}}

\begin{document}

%%%%% =======================================================================
\title{\LARGE %
  Shifted Power Method for\\ H-eigenvalue of Symmetric Tensors}

\institute{%
  \normalsize xxxx xxx \\[1em]
  Department of Mathematics\\
  East China Normal University, Shanghai, China
  \bigskip\bigskip}

\date{SIAM: xxx Conference 20xx \\ xxx, June 20xx}

% ===== title page =====
\begin{frame}[plain]
  \titlepage
\end{frame}

% ===== contents =====
\begin{frame}
  \frametitle{Outline}
   \tableofcontents[hideallsubsections] %[pausesections]
\end{frame}

% ===== main part =====
\section{Tensor and Eigenvalue}
\AtBeginSection[]{\frame{\tableofcontents[currentsection,hideallsubsections]}}

\begin{frame}{Tensor}

The real-valued tensor of order $m$ and dimension $n$
is defined as follows
$$ \A=(a_{i_1\cdots i_m}),\quad a_{i_1\cdots i_m}\in \mathbb{R},
   \quad 1\leq i_1,\cdots,i_m\leq n.
$$

\begin{blist}
\item \myem{Symmetric Tensor}:
 $\A$ is called \myem{symmetric} if its entries do not change
under any permutation of its $m$ indices.\medskip

\item \myem{Nonnegative Tensor}:
$\A=$ is called \myem{nonnegative} if $a_{i_1\cdots i_m} \geq 0 .$
\medskip

\item \myem{Irreducible Tensor \mycite{[CPZ '08]}}:
$\A$ is called \myem{reducible}
if there exists a nonempty proper index subset
$I\subset\{1,2,\ldots,n\}$ such that
$$ a_{i_1\cdots i_m}=0 \quad\text{for all } i_1\in I
  \quad\text{and } i_2,\ldots,i_m\notin I.
$$
If $\A$ is not reducible, then we call $\A$ \myem{irreducible}.
\end{blist}

\end{frame}

%%%%%%%%%%%%%%%%%

\begin{frame}{Tensor-vector product}

Let $r$ be an integer such that $0\leq r\leq m-1$.

\begin{Block}
The $(m-r)$-times product of a symmetric tensor $\A$
with a vector $x$ is denoted by $\A x^{m-r}$
and defined as % \mycite{[KM '11]}
$$
  (\A x^{m-r})_{i_1\cdots i_r}
  := \sum_{i_{r+1},\ldots, i_m}a_{i_1\cdots i_m}
  x_{i_{r+1}}\cdots x_{i_m}
$$
for all $i_1,\ldots,i_r\in\{1,\ldots,n\}$.
\end{Block}

\bigskip

In particular, $\A x^m$ is a scalar and $\A x^{m-1}$ is a vector

\end{frame}


\begin{frame}{Eigenvalue and H-eigenvalue}


\begin{definition} [Eigenvalue, \mycite{[Qi '05, Lim '05]}]
  Let $\A$ be a tensor of order $m$ and dimension $n$.
  Then $\lambda\in\mathbb{C}$ is an eigenvalue of $\A$ if there exists a nonzero vector
  $x\in\mathbb{C}^n$ such that
  $$ % \label{H-eigenpair}
    \A x^{m-1}=\lambda x^{[m-1]},
  $$
  where $x^{[m-1]}=[x_1^{m-1},\cdots,x_n^{m-1}]^T$.
  The vector $x$ is the corresponding eigenvector.
\end{definition}

\bigskip
\subsubtit{H-eigenvalue}
  If, in addition, both $\lambda$ and $x$ are real, then
  they are called the \myem{H-eigenvalue} and \myem{H-eigenvector}, respectively.

\end{frame}



\section{NQZ Algorithm for largest Eigenvalue}
\begin{frame}{NQZ algorithm}

NQZ algorithm \mycite{[NQZ '09]} : an iterative method for finding
the largest eigenvalue of irreducible nonnegative tensors.

\begin{Block}\linespread{1.5}\selectfont
  \begin{algorithmic}[1]
  \State Choose a \myem{positive} vector $x^{(0)}$ and compute
     $y^{(0)}=\A (x^{(0)})^{m-1}$
  \For{$k=1,2,\ldots$, until convergence}

  \State $x^{(k)}=\dfrac{\left(y^{(k-1)}\right)^{[\frac1{m-1}]}}%
                          {\left\|\left(y^{(k-1)}\right)^{[\frac1{m-1}]}\right\|}$

  \State $y^{(k)}=\A (x^{(k)})^{m-1}$

  \State $\lam^-_{k}=\min\limits_{x^{(k)}_i>0}
    \dfrac{y^{(k)}_i}{\left(x^{(k)}_i\right)^{m-1}}$

  \State $\lam^+_{k}=\max\limits_{x^{(k)}_i>0}
    \dfrac{y^{(k)}_i}{\left(x^{(k)}_i\right)^{m-1}}$

  \EndFor
  \end{algorithmic}
\end{Block}

\end{frame}

\begin{frame}{Convergence of NQZ}
  Let $\A$ be an \myem{irreducible nonnegative} tensor of order $m$ and dimension $n$.
  Then
  $$
    \tcbhighmath{\lam^-_k\leq \lam^-_{k+1} \quad\text{and}\quad
     \lim_{k\to\infty} \lam^-_k = \lam^-}
  $$
  and
  $$
    \tcbhighmath{\lam^+_k\geq \lam^+_{k+1} \quad\text{and}\quad
     \lim_{k\to\infty} \lam^+_k = \lam^+}
  $$

  Moreover,
  $$ \tcbhighmath{\lam^- \leq \rho(\A) \leq \lam^+} $$
  where $\rho(\A)$ is the spectral radius  of $\A$.
  % $$ \rho(\A):=\max\{\,|\lam|\ :\ \lam\ \text{is an eigenvalue of } \A\}$$

  \begin{notebox}
    In general, the convergence of NQZ for irreducible nonnegative tensors
    is not guaranteed.
  \end{notebox}

\end{frame}




\section{HOPM Algorithm for Z-eigenvalue}

\begin{frame}{HOPM algorithm}

HOPM \mycite{[LMV '00]} : Higher-Order Power Method
% for \myem{Z-eigenvalue}

\begin{Block}\linespread{1.5}\selectfont
  \begin{algorithmic}[1]
  \State Choose a vector $x^{(0)}\in \R^n$ with $\|x^{(0)}\|=1$
  \State Compute $\lam_0=\A (x^{(0)})^m$
  \For{$k=1,2,\ldots$, until convergence}
  \State $y^{(k)}=\A (x^{(k-1)})^{m-1}$
  \State $x^{(k)}= {y^{(k)}}/{\|y^{(k)}\|}$\smallskip
  \State $\lam_{k}=\A (x^{(k)})^m$
  \EndFor
  \end{algorithmic}
\end{Block}

\begin{blist}
  \item HOPM is proposed for the low-rank tensor approximation
   and predates the definition of tensor eigenvalue problem
  \item The initial vector is not required to be positive
  
  \item HOPM can be used to compute \myem{Z-eigenvalue} of symmetric tensors
\end{blist}

\end{frame}


\begin{frame}{SS-HOPM algorithm}

SS-HOPM \mycite{[KM '11]} : Shifted Symmetric Higher-Order Power Method

\begin{Block}\linespread{1.4}\selectfont
  \begin{algorithmic}[1]
  \State Choose a vector $x^{(0)}\in \R^n$ with $\|x^{(0)}\|=1$
    and a shift $\alpha$
  \State Compute $\lam_0=\A (x^{(0)})^m$
  \For{$k=1,2,\ldots$, until convergence}
  \If{$\al\geq0$}
  \State $y^{(k)}=\A (x^{(k-1)})^{m-1}+\myem{\bm{\alpha x^{(k-1)}}}$
  \Else
  \State $y^{(k)}=-\left(\A (x^{(k-1)})^{m-1}+\myem{\bm{\alpha x^{(k-1)}}}\right)$
  \EndIf
  \State $x^{(k)}=y^{(k)}/\|y^{(k)}\|$
  \State $\lam_{k}=\A (x^{(k)})^m$
  \EndFor
  \end{algorithmic}
\end{Block}

\end{frame}



\section{Shifted Power Method for H-eigenvalue}

\begin{frame}{Power method for H-eigenvalue (S-HOPM-H)}

  $\lambda\in\R$ is an H-eigenvalue if there exists a nonzero vector
  $x\in\R^n$ such that
  $$ \tcbhighmath{\A x^{m-1}=\lambda x^{[m-1]}} $$

  \myem{Symmetric High Order Power Method for H-eigenvalue (S-HOPM-H)}

  \begin{Block}\linespread{1.4}\selectfont
  \begin{algorithmic}[1]
  \State Given a symmetric \myem{even-order} tensor $\A$
  \State choose $x^{(0)}\in \mathbb{R}^n$ with $\|x^{(0)}\|_m=1$
         and compute $\lambda_0=\A \left(x^{(0)}\right)^m$
  \For{$k=1,2,\ldots$, until convergence}
  \State $y^{(k)}=\A \left(x^{(k-1)}\right)^{m-1}$\smallskip
  \State \myem{$z^{(k)}=\left(y^{(k)}\right)^{[\frac{1}{m-1}]}$}
  \State $x^{(k)}=z^{(k)}/\|z^{(k)}\|_m$
  \State $\lambda_{k}=\A \left(x^{(k)}\right)^m$
  \EndFor
  \end{algorithmic}
  \end{Block}
\end{frame}


\begin{frame}{Example 1}

\begin{example}[\mycite{[KR '02]}]
Let $\A$ be a symmetric tensor of order 4 and dimension 3,
whose entries are defined by

\scalebox{0.95}{
$
\begin{array}{llll}
a_{1111}= 0.2883,& a_{1112}= -0.0031, &a_{1113}= 0.1973, &a_{1122}=-0.2485,\\
a_{1123}= -0.2939,& a_{1133}= 0.3847, &a_{1222}= 0.2972, &a_{1223}=0.1862,\\
a_{1233}= 0.0919,& a_{1333}= -0.3619, &a_{2222}= 0.1241, &a_{2223}=-0.3420,\\
a_{2233}= 0.2127,& a_{2333}= 0.2727, &a_{3333}= -0.3054.
\end{array}
$}
\end{example}

\begin{blist}
  \item
  There are totally 27 different eigenvalues, among which 11 are real

\end{blist}

\end{frame}

\begin{frame}{Example}

\renewcommand{\arraystretch}{1.15}
\begin{center}
\begin{table}
\caption{All H-eigenpairs generated by \texttt{Mathematica}}
\centering
\begin{tabular}{r|rrr}\toprule
$\lambda$  &  &   $x^T$ & \\\midrule
2.3129  & [ 0.7875&  0.6483& -0.8138 ]\\
1.9316  & [ 0.8749& -0.6536&  0.6936 ]\\
0.9780  & [ 0.1474& -0.9540&  0.6432 ]\\
0.8944  & [ 0.5223&  0.8048&  0.8434 ]\\
0.7228  & [ 0.8526&  0.4939&  0.8012 ]\\
0.4108  & [ 0.2035& -0.5145& -0.9816 ]\\
0.2528  & [ 1.0000&  0.1020& -0.0868 ]\\
0.2499  & [ 0.4178&  0.9917&  0.2184 ]\\
-0.0887 & [ 0.9158&  0.7376&  0.1559 ]\\
-0.6665 & [ 0.2291& -0.2579&  0.9982 ]\\
-2.6841 & [ 0.7793& -0.8675& -0.5044 ]\\
\bottomrule
\end{tabular}
\end{table}
\end{center}
\end{frame}


\begin{frame}[c]
\frametitle{}
  \begin{center}\Huge\color{red}
    Thank you!
  \end{center}
\end{frame}

\end{document}
